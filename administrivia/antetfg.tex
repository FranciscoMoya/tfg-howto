\section{Preparación del anteproyecto}

Para la planificación emplearemos un \href{https://trello.com}{tablero Trello}. Los tableros Trello permiten agrupar tarjetas en una serie de listas con nombre.  Abre tu propia cuenta y comunica tu usuario al director del TFG para que te comparta el tablero inicial.

En la lista \emph{Recursos} encontrarás una tarjeta denominada \emph{Anteproyecto}.  Ábrela y descarga el adjunto \emph{Anexo TFG-01 Solicitud de asignación de TFG.docx}.  Rellena los datos de los TFG que solicitas y preséntalo en secretaría.

La asignación de los TFG se publica en Campus Virtual.  Cuando te lo hayan asignado vuelve a la misma tarjeta del tablero Trello y descarga el adjunto \emph{Anexo TFG-05 Anteproyecto del TFG.docx}.  Rellena los datos y completa el anteproyecto como se indica a continuación.

\subsection{Estructura del anteproyecto}

La estructura del anteproyecto debe respetar en la medida de lo posible el esquema de la plantilla.  Dedica la mayor parte del tiempo y el espacio a la parte de \emph{Antecedentes}.  No olvides incluir bibliografía relevante específica.  A continuación se incluyen una serie de consejos que debes seguir en cada sección.

\subsubsection{Antecedentes}

Introduce muy brevemente el problema que se pretende abordar y describe de forma muy concisa lo que se ha hecho anteriormente en torno a dicho problema.  Trata de ser sistemático y no olvides destacar las lagunas o carencias de las aproximaciones anteriores.  Si el problema no se ha abordado antes haz una descomposición en subproblemas y analiza cada subproblema.  Los antecedentes pueden incluir también soluciones a problemas análogos.  Usa referencias bibliográficas siempre que sea apropiado.  

Esta sección debe ser la más densa y la más extensa, en torno a un 60\% como mínimo. El objetivo de esta sección es convencer al cliente de que el problema merece la pena y no ha sido resuelto completamente con anterioridad.  Por cierto, el cliente es el que paga.  En el TFG se paga en forma de calificación, así que el cliente es tu director y el tribunal.

No tengas miedo en invertir tiempo en esta sección, todo el trabajo servirá como primera aproximación de los antecedentes en la memoria final. El capítulo 2 de la Plantilla de TFG puede ayudarte a elaborar este apartado, pero recuerda que tienes un límite de 2000 palabras en total.

\subsubsection{Objetivos}

Vuelve al tablero Trello y descarga el anexo de la tarjeta \emph{Plantilla de TFG}.  Lee el capítulo 3.  Los objetivos deben ser SMART y no muy ambiciosos.  Deben estar numerados y pueden descomponerse en subobjetivos si lo consideras necesario.

\subsubsection{Temporización del TFG}

Para estimar correctamente la planificación temporal del proyecto tienes que invertir un tiempo que no tienes.  Con un anteproyecto de 2000 palabras no se tiene una idea lo suficientemente profunda de qué implica el TFG.  Todo lo que pongas en esta sección es pura ciencia ficción.  Así que lo dejo a tu criterio.

Puedes poner (y yo lo recomiendo) una planificación ficticia en cascada (etapas consecutivas) y repartir las 300h del TFG en esas etapas.  También puedes poner una descripción de la metodología que realmente vamos a seguir (ver capítulo 5 de la plantilla de TFG) y planificar temporalmente solamente la primera iteración de dos semanas.  A mi personalmente me da igual, no tengo en cuenta esta sección para valorar el anteproyecto.

\subsubsection{Resultados esperados}

Los resultados esperados tienen que ser tangibles.  En los TFG que yo dirijo los resultados esperados pueden ser de tres tipos, software, hardware o documentos.  Enuméralos indicando el tipo y una breve descripción del resultado esperado.

\subsubsection{Bibliografía}

Lee el anexo A.5 de la \emph{Plantilla de TFG} para aprender cómo se escriben las referencias bibliográficas.

\subsection{Entrega del anteproyecto}

No me lo envíes por correo.  Adjunta el anteproyecto en un comentario de la tarjeta \emph{Anteproyecto}.  Si tienes mucha prisa avísame por teléfono o por correo, pero no me adjuntes el anteproyecto al correo.  En la siguiente reunión presencial te lo puedes llevar firmado.

Este método garantiza que en Trello está la última versión del anteproyecto, que te resultará muy útil como referencia para la ejecución del proyecto.