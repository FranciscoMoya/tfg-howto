\section{Metodología de desarrollo}

En general el TFG está sometido a una serie de incertidumbres, derivadas de tu disponibilidad horaria y de tus conocimientos previos, que hace muy recomendable la aplicación de métodos ágiles.  Los métodos ágiles pretenden precisamente maximizar los resultados en entornos muy dinámicos con alta incertidumbre.

Lee el capítulo \emph{Procedimiento} de la plantilla.  Busca información acerca de \emph{Scrum} para tener una idea general del proceso.  Básicamente se trata de reducir la burocracia al mínimo y, aún así, tener controlado el proceso en todo momento.

Eso significa que no vamos a rellenar ningún anexo TFG-04 (\emph{Informe Reunión Director o Asesor TFG}).  En su lugar, las decisiones quedarán plasmadas en la herramienta Trello y en el sistema de control de versiones.  Considero que es un sistema más transparente, más verificable, y menos burocrático.

\subsection{Control de versiones}
\label{sec:control-versiones}

El control de versiones del proyecto se realizará mediante el uso de \href{https://github.com}{GitHub}.  Abre tu cuenta en 
GitHub y haz un \emph{fork} del repositorio de la plantilla de TFG.  Para ello basta visitar la página \url{https://github.com/FranciscoMoya/eii-tfg} y pinchar en el botón \emph{Fork}, arriba a la derecha.  

En tu nuevo repositorio, añade a mi usuario \emph{FranciscoMoya} como uno de los colaboradores de tu nuevo repositorio.  Para ello pincha en \emph{Settings} arriba y después en \emph{Collaborators}. En la caja de búsqueda escribe \emph{FranciscoMoya} y pincha en el botón de añadir.  Este paso permite que pueda hacer correcciones directamente en el repositorio.  No te preocupes, no editaré archivos sin avisar primero.

Aprovecha para editar todos los datos relevantes en \emph{Settings}.  Cambia el nombre al de tu TFG, y rellena la descripción.  En la página del proyecto pincha en el lapicero que hay junto al texto descriptivo.  Ésto te permitirá editar el archivo \emph{README.md}.  Está en \href{https://github.com/adam-p/markdown-here/wiki/Markdown-Cheatsheet}{\emph{Markdown}}, que es un formato de texto para elaborar documentos simples, como una especie de \LaTeX{} para pobres.  Cuando estés satisfecho pincha en el botón \emph{Commit}.

Tu copia del proyecto tendrá inicialmente solamente la plantilla del proyecto.  No tengas miedo de añadir otro tipo de material producido durante la ejecución del proyecto pero procura evitar archivos generados.  Por ejemplo, no incluyas archivos objeto ni archivos EXE, generados al compilar un programa.  En su lugar, mete solo el código fuente a partir del que se puede generar.

El repositorio es esencial para dormir tranquilos.  He visto al menos dos veces en mi carrera cómo alumnos perdían todo su trabajo por fallo del disco duro.  Procura aprender de los errores de otros.  Si eres disciplinado y guardas tus cambios en el control de versiones yo tendré un mejor seguimiento y tú tendrás la tranquilidad de que hay una gran empresa que hace los backups por tí.  También te libera de estar haciendo copias cuando quieres probar algo nuevo.  No tengas miedo en cambiar cualquier cosa.  Si quieres volver a cualquier estado anterior es fácil.

\subsection{Edición en línea}
\label{sec:edicion-online}

Existe un excelente editor en línea para documentos \LaTeX{} llamado \href{https://overleaf.com}{Overleaf}.  La opción gratuita no incluye la integración con GitHub, por lo que tendremos que hacerla a mano.  Para ello descarga el repositorio GitHub como un archivo comprimido \emph{ZIP}.  Es tan sencillo como ir a la página del proyecto y pinchar en \emph{Clone or download} y posteriormente en \emph{Download ZIP}.

Hazte una cuenta en \href{https://overleaf.com}{Overleaf} y crea un nuevo proyecto a partir del ZIP descargado de GitHub. Para ello pincha en \emph{Nuevo proyecto} y selecciona \emph{Subir proyecto}.  En Overleaf puedes editar cómodamente el documento de la memoria, pero el repositorio GitHub no se actualizará automáticamente.  Volveremos a ésto más adelante.

Pincha en el botón \emph{Share} arriba, y posteriormente en \emph{Turn on link sharing}.  Copia la primera URL que se genera en el tablero Trello como un comentario a la tarjeta \emph{Plantilla de TFG}.  No publiques esta URL en otro sitio.  Cualquiera con acceso a esa URL podría editar tu TFG y generarte verdaderos quebraderos de cabeza.