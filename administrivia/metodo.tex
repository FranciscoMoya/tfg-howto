\section{Metodología de desarrollo}

En general el TFG está sometido a una serie de incertidumbres, derivadas de la disponibilidad horaria del estudiante y de sus conocimientos previos, que hace muy recomendable la aplicación de métodos ágiles.  Los métodos ágiles pretenden precisamente maximizar los resultados en entornos muy dinámicos con alta incertidumbre.

Busca información acerca de \emph{Scrum} para tener una idea general del proceso.  Básicamente se trata de reducir la burocracia al mínimo y, aún así, tener controlado el proceso en todo momento.

Seguiremos una versión simplificada de \emph{Scrum}. Emplearemos iteraciones (\emph{sprints}) de dos semanas.  Usaremos \emph{user stories} no completamente puras, en el sentido que a veces necesitarás aprender o leer, sin repercusión medible en el valor percibido por el \emph{Product Owner}. No te preocupes, pero intenta reducir al mínimo este tipo de \emph{user stories} para tener el proceso controlado.

Puntualmente puede ser necesario una historia de usuario que solo pretende explorar opciones. No hay problema, se denominan \emph{spikes}, pero procura que la exploración no domine tu tiempo dedicado al TFG.

Para la planificación emplearemos un \href{tablero Trello}{https://trello.com}.  Abre tu propia cuenta y comunica tu usuario al director del TFG para que te comparta el tablero inicial.

\subsection{Planificación de sprints}

Añade historias de usuario al Backlog. Puedes añadirlas cuando quieras, pero no reordenes, añade siempre al final. El product owner priorizará las historias. Justo antes de la siguiente iteración tendremos una reunión presencial o virtual para planificar la iteración.

Usando planning poker dimensionaremos las historias de usuario en días de trabajo. El product owner moverá las primeras historias del backlog a ToDo hasta completar los días de la iteración. Los días de trabajo efectivo de la iteración no tienen por qué ser dos semanas, hay que contar con el tiempo realmente dedicado.

Cuando empiece la iteración elige una tarea cualquiera de la lista ToDo y arrástrala a la lista Doing. Cuando termines la tarea arrástrala a QC (quality control). El Scrum Master (yo) revisará que la historia está realmente acabada y si así es la pasará a Done. En caso contrario la pasará a Doing otra vez. Debes estar pendiente y ver los comentarios que se han añadido.

Si en el transcurso de tu trabajo te encuentras con un obstáculo que te impide progresar con una historia, muévela a Blocked.

Cuando acabe el proyecto tendrás todas las historias por orden de terminación en Done. Es una interesante fuente de información para copiar en la memoria de tu TFG. Por tanto tómate en serio la redacción de cada historia de usuario.

Roles
F. Moya: Product Owner, Scrum Master
Alumno: Team Member
Enlaces
Usar trello para Scrum
La guía oficial de Scrum

\section{Scrum para TFG}

\section{Control de versiones}
