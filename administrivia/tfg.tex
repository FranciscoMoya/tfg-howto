\section{Rutina de trabajo}
\label{sec:rutina-tfg}

Recapitulemos brevemente para no perder la visión general.  Tenemos actualmente tres herramientas que sirven a propósitos diferentes.

\begin{itemize}
\item Trello se usa para la información privada del proyecto, el seguimiento y la secuenciación de tareas.  Trello solo lo vemos tú y yo.  Si quieres añadir información sobre el proceso de desarrollo en la memoria procura ir haciendo capturas de pantalla de tu tablero Trello tras cada iteración.
\item Overleaf se usa para editar la memoria.  Tú y yo podemos editarla, y existe una URL para poder ofrecer una vista de solo lectura a otros.  La interfaz es muy intuitiva, así que no deberías tener muchos problemas en usarla.  Revisa el apéndice A de la Plantilla de TFG para tener una idea general de \LaTeX{}.
\item GitHub se usa para tener control de versiones.  De todo, incluída la memoria.  En este capítulo veremos cómo.
\end{itemize}

\subsection{GitHub Desktop}
\label{sec:github-desktop}

El uso del repositorio no es complejo, pero es aún más sencillo si empleamos una herramienta de ayuda, como \href{https://desktop.github.com/}{GitHub Desktop}.  

Instala la aplicación y sigue el tutorial de \href{https://programminghistorian.org/es/lecciones/introduccion-control-versiones-github-desktop}{esta página} para aprender a utilizarla.  No necesitas gran cosa para empezar, puedes parar cuando sepas lo que es un \emph{commit}, cómo se publica con \emph{push}, y cómo se refresca la copia de trabajo con \emph{fetch} o \emph{pull}.  De momento puedes ignorar todo lo relativo a ramas (\emph{branch}), mezclado (\emph{merge}) y \emph{pull requests}.

Los resultados esperados de una historia de usuario van a ser casi siempre un commit en el repositorio.  Es decir, para dar por cerrada una tarea vas a tener que indicarme en un comentario cuál es el commit del repositorio que lo implementa y yo lo voy a comprobar en el propio repositorio.

Cuando la tarea sea de documentación tendrás que usar Overleaf para editar el documento pero luego habrá que trasladarlo a GitHub.  Sigue estos sencillos pasos:

\begin{itemize}
\item Ve a tu proyecto Overleaf y pincha en \emph{Menu} arriba a la izquierda.  Descarga el código fuente del proyecto pinchando en \emph{Source}.  Eso descargará un archivo comprimido ZIP en la carpeta de descargas de tu navegador.  

\item Extrae el contenido del archivo en tu copia local del repositorio.

\item Ejecuta GitHub Desktop y marca para añadir todos los archivos nuevos que hayas añadido a tu proyecto Overleaf.  Rellena una descripción de los cambios y pincha en \emph{Commit}.

\item Propaga los cambios a GitHub pinchando en el botón \emph{Push to origin}.
\end{itemize}

\subsection{Permisos y atribuciones}
\label{sec:permisos-atribuciones}

Si copias en la memoria dibujos, tablas o imágenes de otros, solicita permiso al autor.  En el archivo \texttt{\_FUENTES.txt} escribe el origen de cada archivo conforme los vas añadiendo y anota cuándo y de qué forma solicitaste permiso al autor.  Cuando recibas la autorización guarda la prueba.

Hay una excepción a esta regla.  Cuando la licencia de uso permite expresamente la copia entonces no será necesario solicitar permiso, pero deberás atribuirlo adecuadamente.  Anota la licencia en \texttt{\_FUENTES.txt}.