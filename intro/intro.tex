Este pequeño manual surge de un problema recurrente, muchos alumnos no saben cómo empezar su TFG y una buena parte del tiempo se pierde en adaptarse al método de trabajo.

Has decidido que quieres abordar un TFG en alguna de las líneas que dirigen los autores de este documento.  Bien, es el primer paso, ahora te toca aprender lo básico.  No esperes a presentar el anteproyecto, aprende ahora.  Mejor todavía si estás en el último o penúltimo curso y te planteas realizar el TFG dentro de unos meses.

Es un manual práctico del que solo sacarás algo si lo pones en práctica.  No lo leas sin más, empieza ya tu TFG, aunque no lo tengas decidido. Te servirá como prácticas de lo que debes saber.